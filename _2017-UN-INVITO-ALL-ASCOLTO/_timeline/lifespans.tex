\documentclass[a3paper]{article}
\usepackage[T1,T2A]{fontenc}
\usepackage[utf8]{inputenc}
\usepackage[french,english]{babel}
\usepackage[pdftex]{graphicx}
\usepackage{cmap}
\usepackage{hyperxmp}
\usepackage[landscape,left=0.1cm, right=.1cm, top=0.1cm, bottom=.1cm]{geometry}
\usepackage[hidelinks]{hyperref}
\usepackage[usenames,dvipsnames]{xcolor}
\usepackage{ccicons}
\usepackage{tikz}
\usetikzlibrary{calc}

\begin{document}

%% PDF meta-data
\hypersetup{%
pdftitle={Cosmologia dell'ascolto},
pdfauthor={Giuseppe Silvi},
pdfcopyright={This work is licensed under a Creative Commons Attribution-ShareAlike 3.0 Unported License},
pdfsubject={Cosmologia dell'ascolto},
pdfkeywords={composers,tikz,lifespan,timeline},
pdflicenseurl={http://creativecommons.org/licenses/by-sa/3.0/},
pdfcaptionwriter={Giuseppe Silvi},
pdfcontactcity={Rome},
pdfcontactcountry={Italy},
pdfcontactemail={me@giuseppesilvi.com},
}

%% To put hyperlinks into the nodes (whole area, not only texts)
%% Taken from: http://tex.stackexchange.com/questions/36109/making-tikz-nodes-hyperlinkable
\tikzset{
   hyperlink node/.style={
       alias=sourcenode,
       append after command={
           let     \p1 = (sourcenode.north west),
               \p2=(sourcenode.south east),
               \n1={\x2-\x1},
               \n2={\y1-\y2} in
           node [inner sep=0pt, outer sep=0pt,anchor=north west,at=(\p1)] {\href{#1}{\phantom{\rule{\n1}{\n2}}}}
       }
   }
}

%% Puts new lifespan with description to the diagram
%% Arguments:
%% Arg1: Start year
%% Arg2: End year
%% Arg3: Description
%% Arg4: Color
%% Arg5: URL
%% Optional argument: node parameters (e.g., anchor)
\newcommand{\putLifeSpanDesc}[6][]{%
%% Starting position for the lifespan
\pgfmathsetlengthmacro{\startPos}{(((#2>\constStartYear) ? #2 : \constStartYear) - \constStartYear)*\constMminYear}
%% End position of the lifespan
\pgfmathsetlengthmacro{\endPos}{(((#3<\constStopYear) ? #3 : \constStopYear) - \constStartYear)*\constMminYear}
%% Center position of the lifespan - to put the label there
\pgfmathsetlengthmacro{\centerPos}{(\endPos + \startPos)/2}
%% Width of the lifespan
\pgfmathsetlengthmacro{\spanWidth}{\endPos-\startPos}
% Put the node instead of teh plain rectangle in order to be able to utilize hyperlink node style and make the whole life span clickable
\node[rectangle,rounded corners,fill=#5,inner sep=0pt,minimum height=\constLifeSpanHeight, minimum width=\spanWidth, anchor=south west,hyperlink node={#6}] at (\startPos, 0) {};
%\draw[rounded corners,fill=#5] at (\startPos, 0) rectangle (\endPos, \constLifeSpanHeight);
\node[#1] at (\centerPos, \constLifeSpanHeight + 2mm) {\href{#6}{#4}};
}

%% Puts new lifespan to the diagram
%% Arguments:
%% Arg1: Start year
%% Arg2: End year
%% Arg3: Description
%% Arg4: Color
%% Arg5: URL
%% Optional argument: node parameters (e.g., anchor)
\newcommand{\putLifeSpan}[6][]{%
\putLifeSpanDesc[#1]{#2}{#3}{#4 (#2 -- #3)}{#5}{#6}
}
\newcommand{\putLifeSpanl}[6][]{%
\putLifeSpanDesc[#1]{#2}{#3}{#4 (#2 -- )}{#5}{#6}
}

%% Draws vertical tics for the timeline
%% Arg1: Year
\newcommand{\drawVertTics}[1]{%
%% X pos
\pgfmathsetlengthmacro{\pos}{(#1 - \constStartYear)*\constMminYear}
\draw[ultra thin,densely dotted] (\pos, 1cm) -- (\pos, .9\textheight);
\node at (\pos, 0) {#1};
}

%% Musicians
\colorlet{lifespancolormu}{Purple}
%% Musicians
\colorlet{lifespancolorce}{Red}

%% Philosophers
\colorlet{lifespancolorph}{MidnightBlue}
%% Painters
\colorlet{lifespancolorpa}{ForestGreen}
%% Writers and poets
\colorlet{lifespancolorwr}{Yellow}
%% All others (default)
\colorlet{lifespancolor}{Gray}



%% Updates shift
\newcommand{\updateshift}{\pgfmathsetlengthmacro{\myshift}{\myshift + 1cm}}

\newcommand{\colsign}[1]{\colorbox{#1}{\hspace*{1cm}}}
\newsavebox{\thinkers}
\savebox{\thinkers}{\Large\itshape\parbox{7cm}{%
\begin{itemize}
\item[\colsign{lifespancolormu}] Musicians
\item[\colsign{lifespancolorce}] Musicians
\item[\colsign{lifespancolorph}] Philosophers
\item[\colsign{lifespancolorpa}] Painters
\item[\colsign{lifespancolorwr}] Poets and Prose Writers
\item[\colsign{lifespancolor}] Others
\end{itemize}}}

\noindent\begin{tikzpicture}[x=1mm,y=1mm]
\pgfmathsetlengthmacro{\constLifeSpanHeight}{3mm}
\pgfmathsetmacro{\constStartYear}{1800}%
\pgfmathsetmacro{\constStopYear}{2017}%
\pgfmathsetlengthmacro{\myshift}{1cm}%
%% Scale: points in one year
\pgfmathsetlengthmacro{\constMminYear}{.9\textwidth/(\constStopYear-\constStartYear)}
%% Grid
\foreach \i in {1800,1810,...,2017} {
\drawVertTics{\i}
}
%% Lifespans
\begin{scope}[yshift=\myshift]
\putLifeSpan{1874}{1951}{Arnold Franz Walter Schoenberg}{lifespancolormu}{https://en.wikipedia.org/wiki/Arnold_Schoenberg}
\end{scope}
\updateshift
\begin{scope}[yshift=\myshift]
\putLifeSpan{1923}{1978}{Angelo Maria Ripellino}{lifespancolorwr}{https://it.wikipedia.org/wiki/Angelo_Maria_Ripellino}
\end{scope}
\updateshift
\begin{scope}[yshift=\myshift]
\putLifeSpan{1883}{1945}{Anton Friedrich Wilhelm (von) Webern}{lifespancolormu}{https://en.wikipedia.org/wiki/Anton_Webern}
\end{scope}
\updateshift
\begin{scope}[yshift=\myshift]
\putLifeSpan{1885}{1935}{Alban Maria Johannes Berg}{lifespancolormu}{https://en.wikipedia.org/wiki/Alban_Berg}
\end{scope}
\updateshift
\begin{scope}[yshift=\myshift]
\putLifeSpan{1926}{1980}{Franco Evangelisti}{lifespancolormu}{https://en.wikipedia.org/wiki/Franco_Evangelisti_(composer)}
\end{scope}
\updateshift
\begin{scope}[yshift=\myshift]
\putLifeSpan{1924}{1990}{Luigi Nono}{lifespancolormu}{https://en.wikipedia.org/wiki/Luigi_Nono}
\end{scope}
\updateshift
\begin{scope}[yshift=\myshift]
\putLifeSpan{1818}{1883}{Karl Heinrich Marx}{lifespancolorph}{http://en.wikipedia.org/wiki/Marx}
\putLifeSpan{1912}{1992}{John Cage}{lifespancolormu}{https://en.wikipedia.org/wiki/John_Cage}
\end{scope}
\updateshift
\begin{scope}[yshift=\myshift]
\putLifeSpan{1925}{2016}{Pierre Boulez}{lifespancolormu}{https://en.wikipedia.org/wiki/Pierre_Boulez}
\end{scope}
\updateshift
\begin{scope}[yshift=\myshift]
\putLifeSpanl{1890}{1960}{Boris Leonidovich Pasternak}{lifespancolorwr}{https://en.wikipedia.org/wiki/Boris_Pasternak}
\putLifeSpanl{1977}{2017}{IRCAM}{lifespancolorce}{https://en.wikipedia.org/wiki/Centre_Georges_Pompidou}
\end{scope}
\updateshift
\begin{scope}[yshift=\myshift]
\putLifeSpan{1946}{1970}{Internationale Ferienkurse f\"ur Neue Musik, Darmstadt}{lifespancolorce}{https://en.wikipedia.org/wiki/Darmstadt_School}
\end{scope}
\updateshift\begin{scope}[yshift=\myshift]
\putLifeSpan{1844}{1900}{Friedrich Wilhelm Nietzsche}{lifespancolorph}{http://en.wikipedia.org/wiki/Friedrich_Nietzsche}
\putLifeSpan{1928}{2007}{Karlheinz Stockhausen}{lifespancolormu}{https://en.wikipedia.org/wiki/Karlheinz_Stockhausen}
\end{scope}
\updateshift
\begin{scope}[yshift=\myshift]
\putLifeSpan{1881}{1973}{Pablo Picasso}{lifespancolorpa}{http://en.wikipedia.org/wiki/Picasso}
\end{scope}
\updateshift
\begin{scope}[yshift=\myshift]
\putLifeSpanl{1931}{2017}{Alvin Lucier}{lifespancolormu}{https://en.wikipedia.org/wiki/Alvin_Lucier}
\end{scope}
\updateshift
\begin{scope}[yshift=\myshift]
\putLifeSpanl{1945}{2017}{Giorgio Nottoli}{lifespancolormu}{https://en.wikipedia.org/wiki/Giorgio_Nottoli}
\end{scope}
\updateshift
\begin{scope}[yshift=\myshift]
\putLifeSpanl{1949}{2017}{Alvise Vidolin}{lifespancolormu}{}
\end{scope}
\updateshift
\begin{scope}[yshift=\myshift]
\putLifeSpanl{1953}{2017}{Michelangelo Lupone}{lifespancolormu}{}
\end{scope}
\updateshift
\begin{scope}[yshift=\myshift]
\putLifeSpanl{1956}{2017}{Nicola Bernardini}{lifespancolormu}{}
\end{scope}
\updateshift
\begin{scope}[yshift=\myshift]
\putLifeSpan{1839}{1906}{Paul C\'ezanne}{lifespancolorpa}{http://en.wikipedia.org/wiki/Cezanne}
\putLifeSpan{1912}{1956}{Jackson Pollock}{lifespancolorpa}{http://en.wikipedia.org/wiki/Jackson_Pollock}
\putLifeSpanl{1962}{2017}{Agostino Di Scipio}{lifespancolormu}{}
\end{scope}
\updateshift
\begin{scope}[yshift=\myshift]
\putLifeSpan{1809}{1882}{Charles Darwin}{lifespancolor}{http://en.wikipedia.org/wiki/Charles_Darwin}
\putLifeSpan{1887}{1968}{Marcel Duchamp}{lifespancolorpa}{http://en.wikipedia.org/wiki/Marcel_Duchamp}
\end{scope}

%---------------
%\putLifeSpan{1872}{1970}{Bertrand Russell}{lifespancolorph}{http://en.wikipedia.org/wiki/Bertrand_Russell}
%\putLifeSpan{1724}{1804}{Immanuel Kant}{lifespancolorph}{http://en.wikipedia.org/wiki/Immanuel_Kant}
%\putLifeSpan{1770}{1831}{Georg Wilhelm Friedrich Hegel}{lifespancolorph}{http://en.wikipedia.org/wiki/Hegel}
%\putLifeSpan{1832}{1883}{Édouard Manet}{lifespancolorpa}{http://en.wikipedia.org/wiki/\%C3\%89douard_Manet}
%\putLifeSpan{1856}{1939}{Sigmund Freud}{lifespancolor}{https://en.wikipedia.org/wiki/Sigmund_Freud}
%\putLifeSpan{1889}{1951}{Ludwig Wittgenstein}{lifespancolorph}{http://en.wikipedia.org/wiki/Ludwig_Wittgenstein}
%\putLifeSpan{1903}{1969}{Theodor W. Adorno}{lifespancolorph}{http://en.wikipedia.org/wiki/Adorno}

\node at (55,230) { \usebox{\thinkers} };
\node[font=\Huge\scshape\bfseries,fill=blue!20,rounded corners] at (80, 260) {1850 -- 2020};
\end{tikzpicture}

\bigskip

\centerline{\ccbysa\ \href{http://www.giuseppesilvi.com}{Giuseppe Silvi}, 2017. Conservatorio di Musica S. Cecilia di Roma.}

\end{document}
